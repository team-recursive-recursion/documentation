\documentclass{article}

\title{
    Software Requirements Specification\\
    \begin{large}
        \textit{Golf Course Mapper}
    \end{large}
}
\date{
    \begin{small}
        \today
    \end{small}
}
\author{
    Team Recursive Recursion
}

\begin{document}
    \pagenumbering{gobble}
    \maketitle
    \newpage

    \pagenumbering{arabic}
    \section{Introduction}

    \subsection{Purpose}

    \paragraph{}
    This Software Requirements Specification (henceforth SRS) is intended to be
    a guide on the path of system development. It will aid the developers by
    providing a layout of the project requirements.

    \subsection{Scope}

    \paragraph{}
    The Golf Course Mapper will allow golf course managers to draw the shape of
    their courses using a web application. The owners will map out the details
    of the golf course. This includes mapping the layout of the fairway, green,
    sand and water hazards and details such as the position of the hole and the
    teeing grounds.

    \paragraph{}
    Players currently on the course will then be able to use the provided mobile
    application to view the map of the course and where they are on it
    currently. This has the benefit of allowing players to plan their shots more
    strategically.

    \paragraph{}
    The Golf Course Mapper is not a coaching tool and does not provide a channel
    of communication between the mapper and the player. The web application is
    not to be used by coaches, but solely by managers that wish to map out the
    details of their courses. The Golf Course Mapper is also not a social media
    platform for golfers and will not allow direct communication between
    players.

    \subsection{Definitions, Acronyms \& Abbreviations}
    \begin{itemize}
        \item
            System: Henceforth used to refer to the Golf Course Mapper system
            as a whole.
        \item
            Manager: The manager or owner of a golf course, henceforth used to
            refer to the user of the website.
        \item
            Player: A golf player, henceforth used to refer to the users of the
            mobile application.
        \item
            Website: A web application that is used by golf course managers
            to draw the layout of their golf courses.
        \item
            Mobile App: A native Android mobile app that golf players can use to
            view the golf course.
        \item
            Polygon: The shape drawn on the website to represent a feature of
            the golf course such as the green or the fairway.
        \item
            Back End: The server side that consists of both the DBMS and the
            API.
        \item
            DBMS: Database Management System.
        \item
            API: Application Programming Interface.
    \end{itemize}

    \subsection{Domain Model}

    \paragraph{}
    Refer to figure \ldots, which represents the domain model of the system. At
    the core of the domain model is the \textit{Golf Course} which represents a
    single area where \textit{Players} can go to play the game of golf. Each
    Player is associated with a single Golf Course on which the Player is
    currently playing. Each Golf Course has an assigned \textit{Manager} that is
    solely responsible for mapping and managing the information of the Golf
    Course. Note that a Manager can be assigned to more than one Golf Course.

    \paragraph{}
    Each Golf Course has one or more \textit{Holes} associated with it. A Hole
    represents a single playable map with different \textit{Elements} of spatial
    information. The Elements can be of two different types: \textit{Point}
    Elements and \textit{Polygon} Elements. A Point Element represents
    single-point information of a Hole such as the teeing grounds, location of
    the hole and other points of interest. Polygon Elements represent area
    information of a Hole, such as the area of the rough, fairway, green and
    water and sand hazards.

    \paragraph{}
    A Golf Course may also optionally have Elements that are not associated with
    a specific Hole. This allows the specification of Elements such as points of
    interest or hazardous areas such as bodies of water that may be important to
    more than one Hole. It therefore does not make sense to associate these
    Elements with a specific Hole, but rather with the Golf Course as a whole.

    \newpage

    \section{Overall Description} 

    \subsection{Product Perspective}

    \subsubsection{System Interfaces}
    The System consists of three subsystems, namely the \textit{Mobile App},
    the \textit{Website} and the \textit{Back End}. The Back End additionally
    consists of the \textit{DBMS} and \textit{API} subsystems.

    \subsubsection{User Interfaces}
    \ldots

    \subsubsection{Hardware Interfaces}
    \ldots

    \subsubsection{Software Interfaces}
    \ldots

    \subsubsection{Communication Interfaces}
    \ldots

    \subsubsection{Memory}
    \ldots

    \subsubsection{Operations}
    \ldots

    \subsubsection{Site Adaptation Requirements}
    \ldots

    The product has two user interfaces namely the website and the mobile
    application.\\ The Map that is displayed on the both the website and the
    mobile application is created using the Google Maps API. A custom API was
    created to make database manipulation easier.\\ The mobile application is
    light weight on processing and memory usage since it is only used to view
    the map and do minor calculations such as distance calculation.

    \subsection{Product Functions}
    The main functions are
    \begin{enumerate}
        \item
            Drawing polygons on a map to represent a golf course
        \item
            Displaying the map with the polygons on a mobile device used by
            golfers.
        \item
            Calculate the distance between the player and the hole as well as
            indicating hazards or  obstacles between the player and the hole.
    \end{enumerate}

    \subsection{User Characteristics}
    The product is very simple to use for anyone that has ever had any
    experience working with Google Maps.\\ All icons and user interfaces are
    designed to be as intuitive as possible. This removes the need of a detailed
    tutorial on how to use the system.

    \subsection{Constraints}
    Using a DBMS that simplifies the use of Geographical objects greatly
    improves the development time but it restricts the choice of DBMSs\\ The
    Google Maps API offers 25000 free map loads. After that limit is reached an
    additional fee will be charged for every 1000 loads over the limit.

    \newpage

    \section{Specific Requirements}

    \subsection{Functional Requirements}
    \begin{itemize}
        \item
            The system shall provide a web interface to draw polygons on a map.
        \item
            The system shall be able to draw multiple polygons over one another.
        \item
            The system shall be able to save polygons and their location to a
            database.
        \item
            The system shall be able to retrieve polygons from the database to
            the web interface for further editing.
        \item
            The system shall provide a mobile application to view maps on.
        \item
            The system shall be able to display the user's location on the
            mobile application.
        \item
            The system shall be able to retrieve and display polygons from the
            database on the map.
        \item
            The system shall calculate and display the distance the user is from
            the hole.
        \item
            The system shall display hazards and obstacles on the course between
            the user and the hole.
    \end{itemize}

    \subsection{Performance Requirements}
    \begin{itemize}
        \item
            The system shall be responsive.
        \item
            The system shall use the Google Maps API sparingly to reduce run
            cost.
    \end{itemize}

    \subsection{Architectural Design \& Requirements}
    \ldots (Deployment diagram \& MVC Explanation)

\end{document}
