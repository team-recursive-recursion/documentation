\documentclass{article}

\usepackage{listings}
\usepackage{color}

\definecolor{backcolor}{rgb}{0.95, 0.95, 0.95}

\lstdefinestyle{codestyle} {
    backgroundcolor=\color{backcolor}
}

\lstset{style=codestyle}

\title{
    Coding Standards\\
    \begin{large}
        \textit{Golf Course Mapper}
    \end{large}
}
\date{
    \begin{small}
        \today
    \end{small}
}
\author{
    Team Recursive Recursion
}

\begin{document}
    %===========================================================================
    % TITLE
    %===========================================================================
    \pagenumbering{gobble}
    \maketitle
    \newpage

    %===========================================================================
    % DOCUMENT
    %===========================================================================
    \pagenumbering{arabic}

    \section{File Standards}
    \label{sec:fs}

    \subsection{Format}
    \label{sec:fs-format}

    \begin{itemize}
        \item Files are encoded using the \textbf{UTF-8} character set.
        \item Lines should not be longer than \textbf{80 columns}.
        \item \textbf{Soft tabs} equating to \textbf{4 spaces} should be used.
        \item Each level of indentation uses \textbf{1 tab}.
        \item Continuation indentation uses \textbf{2 tabs}.
        \item Every file contains a \textbf{file header} (described below).
    \end{itemize}

    \subsection{Headers}
    \label{sec:fs-headers}

    \paragraph{}
    Headers are always on the first line of a file and are placed in comments.
    See section \ref{sec:ls} for language-specific header styling. Each header
    contains the following information:

    \begin{itemize}
        \item Name of the file
        \item Author of the file
        \item Name of the class(es) contained within the file (if any)
        \item Short description of the file
    \end{itemize}

    Header files are structured as the following:

    \begin{lstlisting}[language=Java]
Filename: File.ext
Author  : John Doe
Class   : SampleClass

    The SampleClass contains many different sample
    methods.
    \end{lstlisting}

    \newpage

    \section{Language Standards}
    \label{sec:ls}

    \subsection{Java, C\# \& TypeScript}
    \label{sec:ls-java}

    \subsubsection{Naming Conventions}
    \label{sec:ls-java-nc}

    \begin{itemize}
        \item \textbf{Variables} are named using camel casing. Descriptive names
                should be used with the exception of counters in loops.
        \item \textbf{Member variables} are named similarly to regular variables
                with the addition of an \textit{underscore} prefix.
        \item \textbf{Classes} start with a \textit{capital} letter and use
                camel casing.
        \item \textbf{Functions} are also named similar to regular variables and
                should be descriptive.
        \item \textbf{Recursive functions} are named similarly to regular
                functions, but have a \texttt{\_r} postfix to the name.
    \end{itemize}

    \begin{lstlisting}[language=java]
class SampleClass {

    int _someMember;

    public void myFunction() {
        int someInteger;
    }

    public void myFunction_r() {
        myFunction_r();
    }
}
    \end{lstlisting}

    \subsubsection{Style}
    \label{sec:ls-java-st}

    \paragraph{}
    \textbf{Braces} will be styled in the following manner:
    \begin{itemize}
        \item Opening braces are placed on the same line as the header.
        \item Closing braces are placed on a separate line at the same
                indentation level as the header.
        \item Else clauses are placed on the same line as the closing brace.
        \item While clauses of a do-while are placed on the same line as the
                closing brace.
        \item Braces are never left out for one-line loops or conditions.
    \end{itemize}

    \begin{lstlisting}[language=Java]
if (condition) {
    statement;
} else {
    statement;
}

while (condition) {
    statement;
}

do {
    statement;
} while (condition);

class SampleClass {
    public void method() {
        statement;
    }
}
    \end{lstlisting}

    \textbf{Continuation lines} should end on the operator as to indicate that
    the line is not complete and has a continuation.

    \begin{lstlisting}[language=Java]
double result = example + of + (a * very) / 
        long - equation;
    \end{lstlisting}

    \subsubsection{Comments}
    \label{sec:ls-java-com}

    \paragraph{}
    \textbf{File headers} are structured according to section
    \ref{sec:fs-headers} and are styled as the following:

    \begin{lstlisting}[language=Java]
/***
 * Filename: SampleClass.java
 * Author  : John Doe
 * Class   : SampleClass
 *
 *     The SampleClass contains many different sample
 *     methods.
 ***/
    \end{lstlisting}

    \textbf{Function headers} are provided for every function and take the
    following form (the description can be omitted in the case of simple
    functions, such as mutators \& accessors).

    \begin{lstlisting}[language=Java]
/***
 * function(ParType1, ParType2) : ReturnType
 *
 *     Description of the function.
 ***/
ReturnType function(ParType1 p1, ParType2 p2) {
    ...
}


/***
 * SampleClass(ParType1, ParType2) <<constructor>>
 *
 *     Description of the constructor.
 ***/
SampleClass(ParType1 p1, ParType2 p2) {
    ...
}
    \end{lstlisting}

    \textbf{Inline comments} should be kept to a minimum, since code should be
    self-documenting. Only use inline comments in the case of code that may be
    difficult to understand.

    \subsection{HTML \& CSS}
    \label{sec:ls-html}

    \subsubsection{Style}
    \label{sec:ls-html-st}

    \paragraph{}
    \textbf{CSS} should be styled as follows:

    \begin{lstlisting}
selectors {
    some-attribute : style;
    another-attribute : style;
}

more {
    some-attribute : style;
}
    \end{lstlisting}

    \textbf{HTML style} follows the following rules:

    \begin{itemize}
        \item Opening and closing tags of \textbf{block} elements should be kept
            on their own lines, with the content indented.
        \item Opening and closing tags of \textbf{inline} elements should be
            kept on the same line, with the content between the tags.
    \end{itemize}

    \begin{lstlisting}[language=html]
<div>
    <p>
        Here is some <span class="red">red</span> text!
    </p>
</div>
    \end{lstlisting}

    \subsubsection{Comments}
    \label{sec:ls-html-com}

    \paragraph{}
    \textbf{File headers} are structured according to section
    \ref{sec:fs-headers} and are styled for CSS and HTML, respectively, as
    the following:

    \begin{lstlisting}[language=java]
/***
 * Filename: style.css
 * Author  : John Doe
 *
 *     The styling for some example page is contained
 *     here and applies a material style.
 ***/
    \end{lstlisting}

    \begin{lstlisting}[language=html]
<!--
    Filename: page.html
    Author  : John Doe
    
        The page displays some content.
-->
    \end{lstlisting}

    \newpage

    \section{Repositories Structure}
    \label{sec:rs}

    \subsection{Git Repositories}
    \label{sec}

    \paragraph{}
    There are four (4) repositories that are used for the project:
    \texttt{web-app}, \texttt{mobile-app}, \texttt{mapper-api} and
    \texttt{documentation}.

    \begin{itemize}
        \item \texttt{web-app}
            \subitem The web application used for mapping golf courses.
        \item \texttt{mobile-app}
            \subitem The Android application used for viewing golf courses.
        \item \texttt{mapper-api}
            \subitem The API used to access the database.
        \item \texttt{documentation}
            \subitem All documentation relating to the project, including this
                    file.
    \end{itemize}

    \subsection{File Structure}

    \paragraph{}
    Each repository has a \texttt{README.md}, \texttt{LICENSE.txt} and a
    \texttt{.gitignore} file in the root of the repository.

    \paragraph{}
    \texttt{mapper-api} and \texttt{web-app} contain a folder that contains
    the source code of the project. The folder is organised according to the
    standard template of \textit{ASP.NET Core} web applications.

    \paragraph{}
    \texttt{mobile-app} contains a folder that contains the source code of the
    project. The folder is organised according to the standard template of
    \textit{Android Studio} projects.

    \paragraph{}
    \texttt{documentation} contains a folder for each of the four (4) documents.
    These folders contain the \texttt{.tex} source files of the documents, as
    well as any auxiliary files that are needed by the document. In addition,
    there is a \texttt{publish} folder that contains the final PDF versions of
    each document, and a \texttt{other} folder that contains extra documents
    that do not form part of one of the four (4) main documents.


    % TODO git standards & code review
    %\newpage

    %\section{Git Standards}
    %\label{sec:gs}

    %\newpage

    %\section{Code Review}
    %\label{sec:cr}

\end{document}
